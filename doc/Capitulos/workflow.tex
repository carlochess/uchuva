\section{Workflow}

En computación científica, es poco probable que un solo \emph{job} vaya a ser ejecutado para obtener los resultados de una investigación. Lo común es que se ejecuten numerosos \emph{jobs} entre los cuales puede haber dependencias, es decir, donde los datos de salida de uno sean los datos de entrada de otro, o bien, que dos procesos no puedan ser ejecutados al mismo tiempo porque no estan disponibles los suficientes recursos para llevar acabo la operación. Estas relaciones pueden ser modelado haciendo uso de un workflow.\\

Un \emph{workflow} puede ser usado tanto en una empresas como en un ámbito académico, paro ello se distinguen dos definiciones, en el ámbito científico se denominan “Scientific Workflow” y se convierten en la representación de un conjunto de procesos que tienen relaciones de precedencia y se agrupan entre sí para realizar un trabajo \cite{qin2012scientific}. Mientras que en el ámbito empresarial, se denominan ``Business Workflow'' y se definen como “El proceso de automatización de un todo o de una parte durante el cual documentos, información o tareas son transferidas de un participante a otro de acuerdo a una serie de reglas”[5]. Para ambos tipos de workflow, este ofrecen ventajas como:

\begin{itemize}
\item La Descripción de procedimientos en procesos
\item La coordinación de procesos y datos
\item La automatización de procesos y datos
\item El rastreo de la procedencia de procesos y datos
\end{itemize}

Pero también trae consigo problemas como una baja tolerancia a fallos debido a un pobre uso de estructuras de control y excepciones, una limitada debido a que se debe asumir que todas las partes del workflow pueden fallar y ya que no hay excepciones, una vez es detectado el fallo la única solución es iniciar la ejecución desde la parte donde ocurrió el error, conociéndose esto como un workflow de recuperación, sin que el workflow mismo pueda recuperarse porque acude a otros mecanismos. Por último, si un workflow esta compuesto por una gran cantidad de nodos, realizar un debug durante o después de su ejecución se vuelve una tarea difícil \cite{WorkflowGrid:2006}.\\

\subsection{Workflow científicos y Workflow de negocio}

Desde la revolución industrial, las empresas han automatizado la mayoría de sus procesos de negocio y fue gracias a la revolución de las tecnologías de la información que hubo una mejora sustancial en el manejo de los datos. Un área encargada de la modelación de procesos dentro de las empresas se denomina BPM (Business process model) y fue tanto el éxito que en 1993 se consolidó el Workflow Management Coalition, una entidad de adoptantes, desarrolladores, consultantes y analistas para la mejora masificación del concepto de workflow. Su importancia radica en que puede ser una serie de tareas documentadas, frecuentemente aplicados como una parte de un estándar de la empresa y apoyandose en las tecnologías de la información y los computadores.\\

Los Workflows científicos y empresariales pueden ser clasificados  \cite{LeymanRoller} a su vez en cuatro tipos:

\begin{itemize}
\item Ad Hoc: aquellos que son creados usando una metodología poco formal y son ejecutados una sola vez.
\item Administrativos: aquellos que se crean frecuentemente pero que  no están supeditados a la negocio central de la empresa (finanzas, contaduría)
\item Repetitivos: aquellos que son creados y ejecutados en múltiples ocasiones y que posiblemente se vuelven parte de una práctica o un estándar.
\item Colaborativos: Aquellos que tienen un valor grande para la empresa y que además incluye varios sub workflows.
\end{itemize}

Una preocupación en algunos de los workflow de negocio se centra en la seguridad e integridad de cada una de las tareas que lo componen porque es necesario asegurar que termine y que sea exitoso. Esta preocupación puede ser solucionada haciendo uso de transacciones ACID. Pero por otro lado, los workflow científicos prescinden de esta preocupación puesto a que un experimento puede ser exitoso o no, puede culminar o no, asi que es poco probable hacer un roll back en un experimento científico. \\

En ciencias, es deseable que un experimento sea reproducible y repetible por la comunidad científica ya que este es un pilar del método científico y permite a otros investigadores cerciorarse que los resultados corresponden a la consecuencia de una serie de pasos realmente elaborados. Mientras que para algunas empresas, estos requerimientos no son si quiera considerados\\

Los workflow científicos surgen de la necesidad de resolver el problema de la complejidad en la ejecución de múltiples aplicaciones científicas durante un experimento. Estos ofrecen una forma declarativa de modelar y especificar qué se debe lograr con el resultado, prescindiendo del cómo o dónde debe ser ejecutado. Las aplicaciones de software tiene funcionalidades  y exponen interfaces a sus funcionalidades,  por tanto es necesario unir dichas funcionalidades entre los softwares existentes. La forma de controlar la complejidad que esto representa, es la composición, que es la forma en la que se construyen software más complejos a partir de algunos preexistentes. El control efectivo de la complejidad es en realidad el acto de crear softwares que pueden ser combinados de esta forma incluso desde su forma más básica como lo es la composición de funciones \cite{wforescience:2007}. \\

Por último, los workflows científicos tienden a cambiar rápidamente mientras que los workflows empresariales no son tan dinámicos y tienden a consolidarse. \\

\subsection{Herramientas creadas}

Numerosas aplicaciones han sido implementadas y publicadas para la solución de problemas particulares, por ejemplo, para la ejecución de múltiples Jobs en HTCondor, entre los cuales hay dependencias entre sí, se puede usar un programa llamado Dagman (creado por el mismo equipo de HTCondor). Este programa usa la descripción de múltiples tareas junto con cada una de las dependencias, para formar un grafo acíclico dirigido. El objetivo del DAGMan es el de automatizar el envío y la gestión de workflows complejos, centrado en la fiabilidad y la tolerancia a fallos de cara a una variedad de errores. De hecho, Dagman puede reanudar la ejecución de un Dag cuyo nodo haya fallado. \\

Junto a Dagman también se encuentra Pegasus, un administrador de workflows que permite ejecutarlos en infraestructuras no homogéneas, que varían desde una simple workstation hasta una nube pública usando HTCondor, Globus o Amazon WS. Además, permite monitorear el estado del workflow a través de un dashboard web. \\

Tavera es un sistema open-source para la diseño y la gestión de workflows científicos. Ha sido usado en bioinformática y física. Esta basado en el lenguaje XScufl (XML  Simple   Conceptual  Unified  Flow) y está implementado en java. Se le considera un manejador de workflows de datos que consume, a través de webservices soap, los recursos que exponen numerosas organizaciones.\\

Para tal fin también se ha creado Triana, un software opensource que combina una interfaz gráfica para la modelación de workflows con herramientas para el análisis de datos. Permite el uso de ciclos dentro del workflow. \\

%“Programming Scientific and Distributed Workflow with Triana  Services”
%http://www.trianacode.org/docs/looping%20tutorial.pdf

Más allá de las plataformas, también se han creado lenguajes para la modelación de workflows, uno de ellos es denominado Swift-lang en el cual se pueden expresar, mediante scripts, la ejecución concurrente de programas ordinarios. Pero además se han creado lenguajes de programación de propósito múltiple que permiten ejecutar varias instancias del mismo programa haciendo uso de un cluster, tal es el caso de Julia, un lenguaje creado en el MIT con una sintaxis similar a Matlab y enfocado a uso en computación científica o de alto rendimiento.\\
%https://github.com/common-workflow-language/common-workflow-language

Para workflows empresariales también se han creado numerosos framework enfocados a datos, entre ellos Luigi, azkaban, airflow, etc. Pero tambien para modelar procesos de negocio como Bonita BPM. Tambien resaltan los software para la construcción de software Make, Rake, Drake entre otros.
%https://www.michaelcho.me/article/data-pipelines-airflow-vs-pinball-vs-luigi
%http://www.commonwl.org/
%https://github.com/pditommaso/awesome-pipeline
%Aplicaciones: https://arxiv.org/pdf/0808.3545.pdf

\subsection{Dag}
Un subconjunto de workflows pueden ser representados a través de grafos acíclicos directos (Dag), un grafo finito dirigido sin ciclos el cual consta de un número finito de nodos y aristas. Para este contexto, los workflow o dags agrupan un conjunto de jobs (en nodos) interdependientes (de acuerdo a sus aristas) para los cuales aseguran su culminación impidiendo que un ciclo se encuentre presente. Además, aprovechan la propiedad en la que un workflow puede ser ordenado topológicamente (los vértices son ordenados de tal forma que el vértice inicial de una arista va primero que el vértice final de la misma) para determinar qué tarea o grupo de tareas deben iniciar la ejecución. Como se ha mencionado, prescinden de estructuras de control y ciclos.\\

Un dag exhibe cuatro tipo de subestructuras o subgrafos:
\begin{itemize}
\item secuencial: un grupo de tareas que, ordenadas parcialmente tiene dependencias (directas o indirectas).
paralela: un grupo de tareas las cuales no dependen entre sí. Esta estructura es conocida en paralización como “embarrassing parallelism”, ya que es la forma más simple de paralelizar la ejecución de un programa
\item fork: un grupo de tareas que dependan de una tarea.
\item join: una tarea que dependa de un grupo de tareas.
\end{itemize}

Estas subestructuras pueden ser usadas por un agendador para tomar decisiones tales como cuáles tareas pueden ejecutarse al tiempo o cuales deben ejecutarse antes de otras.
% http://www.cloudbus.org/papers/workflow-sigmod05.pdf
