
\section{Trabajo Futuro}

\begin{itemize}
\item Se podría desarrollar un módulo que agenda mejor el envió de los nodos del DAG (acortar ramas, agregue condicionales).
\item Se debe ampliar las características y funcionalidades del modelamiento en el workflow. Por ejemplo, el poder representar gráficamente los archivos de entrada y salida.
\item Se debe mejorar el sistema de archivos integrándolo con ftp.
\item Mejorar el trabajo en equipos - Compartir archivos y dags.
\item Mejorar la arquitectura de la aplicación - Microservicios.
\item Crear una librería para la generación de archivos así como la tesis lo hace con los programas - como formularios.
\item Mejorar el manejo de errores.
\end{itemize}

\section{Conclusiones}

\begin{itemize}
\item El uso de un portal web visibiliza los servicios ofrecidos por el cluster
\item Utilizar herramientas de visualización permite conocer el funcionamiento en distintos ámbitos
\item En el prototipo funcional se desarrollaron funcionalidades que permiten administrar la ejecución de los jobs en distintos administradores de tareas
\item Lo workflows son una herramienta cómoda para modelar soluciones a problemas en el ámbito científico
\end{itemize}