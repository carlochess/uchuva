\section{Introducción}


La curva de aprendizaje que necesita una persona para usar una aplicación científica es muy elevada, aun cuando se consigue manejar con cierta destreza una herramienta, una incorrecta administración de los recursos computacionales conlleva a pérdidas de dinero y de tiempo. \\

Con el uso de un portal web accesible y seguro que agrupe y exponga una interfaz amigable y simple, el usuario puede minimizar o reducir los tiempos asociados al aprendizaje y uso de la aplicación. Esto resulta en una disminución de los costos y la posibilidad de dedicarle mas tiempo a otros menesteres del proyecto, como por ejemplo el análisis de resultados o la escritura del reporte. \\

Tambien, algo que resulta difícil para los nuevos usuarios es la existencia de una gran cantidad de aplicaciones científicas que ofrecen únicamente una interfaz por consola de comandos \cite{Shneiderman:1983}, pero ahora con la masificación del internet en  hogares y el masivo uso de los computadores personales, se pueden crear interfaces amigables desde un navegador desde las cuales se pueda acceder a una aplicación ocultando el hecho de estar usando una línea de comandos.\\

Ademas, se puede acceder a una vasta red de computadores y desde allí, con los permisos pertinentes, se pueden usar varias de las aplicaciones y los recursos provistos.\\

Este trabajo de grado pretende explorar y usar algunas técnicas y estrategias en la administración de recursos computacionales para así incentivar el uso de herramientas científicas en personas que recién empiezan a usarlas.
